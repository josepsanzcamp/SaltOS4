\documentclass[a4paper]{article}
\usepackage{graphicx}
\usepackage{paralist} % needed for compact lists
\usepackage[normalem]{ulem} % needed by strike
\usepackage[urlcolor=myblue,colorlinks=true,linkcolor=myblue]{hyperref}
\usepackage[english]{babel}
\usepackage{ucs}
\usepackage[utf8x]{inputenc}
\usepackage{eurosym}
\usepackage{sans}
\usepackage{fullpage}
\usepackage{listings}
\usepackage{xcolor}
\usepackage{sectsty}
\allsectionsfont{\color{myblue}}
\definecolor{myblue}{RGB}{39,128,227}
\setlength{\parindent}{0mm}
\setlength{\parskip}{3mm}
\setlength{\plparsep}{2.5mm}
\def\htmladdnormallink#1#2{\href{#2}{#1}}
\definecolor{mygrey}{rgb}{0.9,0.9,0.9}
\usepackage{courier}
\lstset{basicstyle=\ttfamily,backgroundcolor=\color{mygrey},breaklines=true}
\usepackage{tocloft}
\setlength{\cftsubsubsecnumwidth}{13mm}
\setlength\cftparskip{3mm}


\title{Devel Documentation}
\author{SaltOS 4.0 r1930}
\begin{document}
\date{April 2025}
\maketitle
\clearpage

\tableofcontents
\clearpage


\hypertarget{toc1}{}
\section{Introduction}

SaltOS4 is a modular, extensible framework for building Rich Internet Applications (RIAs). It is designed for rapid development of dynamic, offline-capable applications using a clean separation between backend and frontend.

It is composed of:

\begin{compactitem}
\item[\color{myblue}$\bullet$] A PHP backend (`api/`)
\item[\color{myblue}$\bullet$] A JavaScript frontend (`web/`)
\item[\color{myblue}$\bullet$] REST/JSON API with support for CLI execution
\item[\color{myblue}$\bullet$] Offline operation through a Service Worker proxy
\item[\color{myblue}$\bullet$] Declarative and dinamic app definitions via XML and/or YAML
\item[\color{myblue}$\bullet$] Shared templates and logic to reduce boilerplate
\item[\color{myblue}$\bullet$] Clean separation of frontend and backend
\item[\color{myblue}$\bullet$] Rapid development of apps with shared templates and logic
\end{compactitem}


\hypertarget{toc2}{}
\section{Backend Structure \& API (api/)}

The backend is under the `api/` folder and contains four folders (autoload, database, lib and actions) to organize the code depending their usage and functionality.

\hypertarget{toc3}{}
\subsection{Autoloaded Modules (`php/autoload/`)}

Core functions automatically loaded on each request:

\begin{compactitem}
\item[\color{myblue}$\bullet$] `autoload/apps.php`: * Apps helper module
\item[\color{myblue}$\bullet$] `autoload/array.php`: * Array helper module
\item[\color{myblue}$\bullet$] `autoload/compat.php`: * Compatibility helper module
\item[\color{myblue}$\bullet$] `autoload/config.php`: * Config helper module
\item[\color{myblue}$\bullet$] `autoload/database.php`: * Database helper module
\item[\color{myblue}$\bullet$] `autoload/datetime.php`: * Datetime helper module
\item[\color{myblue}$\bullet$] `autoload/error.php`: * Error helper module
\item[\color{myblue}$\bullet$] `autoload/exec.php`: * Execution helper module
\item[\color{myblue}$\bullet$] `autoload/file.php`: * File utils helper module
\item[\color{myblue}$\bullet$] `autoload/getdata.php`: * Get data helper module
\item[\color{myblue}$\bullet$] `autoload/gettext.php`: * Gettext helper module
\item[\color{myblue}$\bullet$] `autoload/iniset.php`: * Iniset helper module
\item[\color{myblue}$\bullet$] `autoload/json.php`: * Json helper module
\item[\color{myblue}$\bullet$] `autoload/log.php`: * Log helper module
\item[\color{myblue}$\bullet$] `autoload/memory.php`: * Memory helper module
\item[\color{myblue}$\bullet$] `autoload/mime.php`: * Mime helper module
\item[\color{myblue}$\bullet$] `autoload/output.php`: * Output helper module
\item[\color{myblue}$\bullet$] `autoload/pcov.php`: * PCOV helper module
\item[\color{myblue}$\bullet$] `autoload/perms.php`: * Permissions helper module
\item[\color{myblue}$\bullet$] `autoload/random.php`: * Random helper module
\item[\color{myblue}$\bullet$] `autoload/semaphores.php`: * Semaphore helper module
\item[\color{myblue}$\bullet$] `autoload/server.php`: * Server helper module
\item[\color{myblue}$\bullet$] `autoload/sql.php`: * SQL utils helper module
\item[\color{myblue}$\bullet$] `autoload/strings.php`: * String utils helper module
\item[\color{myblue}$\bullet$] `autoload/system.php`: * System helper module
\item[\color{myblue}$\bullet$] `autoload/tokens.php`: * Tokens helper module
\item[\color{myblue}$\bullet$] `autoload/user.php`: * User helper module
\item[\color{myblue}$\bullet$] `autoload/version.php`: * Version helper module
\item[\color{myblue}$\bullet$] `autoload/xml2array.php`: * XML to Array helper module
\item[\color{myblue}$\bullet$] `autoload/yaml.php`: * Yaml helper module
\item[\color{myblue}$\bullet$] `autoload/zindex.php`: * Main execution module
\end{compactitem}

\hypertarget{toc4}{}
\subsection{Database Drivers (`php/database/`)}

Supports:

\begin{compactitem}
\item[\color{myblue}$\bullet$] `database/libsqlite.php`: * SQLite3 functions library
\item[\color{myblue}$\bullet$] `database/mysqli.php`: * MySQL improved driver
\item[\color{myblue}$\bullet$] `database/pdo\_mssql.php`: * PDO MsSQL driver
\item[\color{myblue}$\bullet$] `database/pdo\_mysql.php`: * PDO MySQL driver
\item[\color{myblue}$\bullet$] `database/pdo\_sqlite.php`: * PDO SQLite driver
\item[\color{myblue}$\bullet$] `database/sqlite3.php`: * SQLite3 driver
\end{compactitem}

\hypertarget{toc5}{}
\subsection{Libraries (`php/lib/`)}

Not autoloaded; provide extra functionality:

\begin{compactitem}
\item[\color{myblue}$\bullet$] `lib/actions.php`: * Actions module
\item[\color{myblue}$\bullet$] `lib/array2xml.php`: * Array to XML helper module
\item[\color{myblue}$\bullet$] `lib/ascii.php`: * Make Table ASCII
\item[\color{myblue}$\bullet$] `lib/auth.php`: * Login functions
\item[\color{myblue}$\bullet$] `lib/barcode.php`: * Barcode helper module
\item[\color{myblue}$\bullet$] `lib/browser.php`: * Browser helper module
\item[\color{myblue}$\bullet$] `lib/captcha.php`: * Captcha helper module
\item[\color{myblue}$\bullet$] `lib/color.php`: * Color helper module
\item[\color{myblue}$\bullet$] `lib/control.php`: * Control helper module
\item[\color{myblue}$\bullet$] `lib/cron.php`: * Cron utils helper module
\item[\color{myblue}$\bullet$] `lib/dbschema.php`: * Database schema helper module
\item[\color{myblue}$\bullet$] `lib/depend.php`: * Dependencies feature
\item[\color{myblue}$\bullet$] `lib/export.php`: * Export helper module
\item[\color{myblue}$\bullet$] `lib/files.php`: * Files module
\item[\color{myblue}$\bullet$] `lib/gc.php`: * Garbage collector helper module
\item[\color{myblue}$\bullet$] `lib/gdlib.php`: * GD utils helper module
\item[\color{myblue}$\bullet$] `lib/geoip.php`: * GeoIP helper module
\item[\color{myblue}$\bullet$] `lib/help.php`: * Help feature
\item[\color{myblue}$\bullet$] `lib/import.php`: * Import file helper module
\item[\color{myblue}$\bullet$] `lib/indexing.php`: * Make index helper module
\item[\color{myblue}$\bullet$] `lib/log.php`: * Log helper module
\item[\color{myblue}$\bullet$] `lib/math.php`: * Math utils helper module
\item[\color{myblue}$\bullet$] `lib/notes.php`: * Notes module
\item[\color{myblue}$\bullet$] `lib/password.php`: * Password helper module
\item[\color{myblue}$\bullet$] `lib/pdf.php`: * PDF helper module
\item[\color{myblue}$\bullet$] `lib/push.php`: * Push utils helper module
\item[\color{myblue}$\bullet$] `lib/qrcode.php`: * QRCode helper module
\item[\color{myblue}$\bullet$] `lib/score.php`: * Score image helper module
\item[\color{myblue}$\bullet$] `lib/security.php`: * Security helper module
\item[\color{myblue}$\bullet$] `lib/setup.php`: * Setup helper module
\item[\color{myblue}$\bullet$] `lib/trash.php`: * Send file to trash
\item[\color{myblue}$\bullet$] `lib/unoconv.php`: * Unoconv library
\item[\color{myblue}$\bullet$] `lib/upload.php`: * Add upload file
\item[\color{myblue}$\bullet$] `lib/version.php`: * Version helper module
\end{compactitem}

\hypertarget{toc6}{}
\subsection{Action Modules (`php/action/`)}

Handle concrete system actions (CLI-aware where needed):

\begin{compactitem}
\item[\color{myblue}$\bullet$] `action/add.php`: * Add log action
\item[\color{myblue}$\bullet$] `action/app.php`: * Application action
\item[\color{myblue}$\bullet$] `action/auth.php`: * Authentication helper module
\item[\color{myblue}$\bullet$] `action/cron.php`: * Garbage Collector action
\item[\color{myblue}$\bullet$] `action/gc.php`: * Garbage Collector action
\item[\color{myblue}$\bullet$] `action/image.php`: * BarCode action
\item[\color{myblue}$\bullet$] `action/indexing.php`: * Make indexing action
\item[\color{myblue}$\bullet$] `action/integrity.php`: * Make indexing action
\item[\color{myblue}$\bullet$] `action/ping.php`: * Ping action
\item[\color{myblue}$\bullet$] `action/push.php`: * Garbage Collector action
\item[\color{myblue}$\bullet$] `action/setup.php`: * DB Schema action
\item[\color{myblue}$\bullet$] `action/upload.php`: * Add files action
\end{compactitem}

\hypertarget{toc7}{}
\subsection{Other API Components}

\begin{compactitem}
\item[\color{myblue}$\bullet$] `index.php`: entry point that loads autoload and then `zindex.php`
\item[\color{myblue}$\bullet$] `img/`: SaltOS logos and related branding
\item[\color{myblue}$\bullet$] `locale/`: multilingual resources (`.yaml`, `.odt`, `.pdf`)
\item[\color{myblue}$\bullet$] `xml/`: base configuration
  \begin{compactitem}
  \item[\color{myblue}$\bullet$] `config.xml`: contains the global system configuration, including general options, paths, preferences, and base parameters that affect all of SaltOS4.
  \item[\color{myblue}$\bullet$] `cron.xml`: defines the scheduled tasks that the system must run automatically, such as data cleanup, email sending, or synchronizations.
  \item[\color{myblue}$\bullet$] `locale.xml`: specifies the available languages in the system and their codes, allowing the interface localization to be managed.
  \item[\color{myblue}$\bullet$] `bs\_theme.xml`: defines the visual themes compatible with Bootstrap that the system can use to customize the interface appearance.
  \item[\color{myblue}$\bullet$] `css\_theme.xml`: contains additional visual style definitions, such as colors, fonts, and CSS rules to adapt the system's look and feel.
  \item[\color{myblue}$\bullet$] `dbschema.xml`: describes the general database schema — tables, columns, indexes, and relationships required for the overall functioning of SaltOS4.
  \item[\color{myblue}$\bullet$] `dbstatic.xml`: includes static data that must be inserted into certain tables during system initialization, such as user types, default values, or base configurations.
  \end{compactitem}
\end{compactitem}

\hypertarget{toc8}{}
\subsection{API Access}

SaltOS4 supports access via HTTP or CLI.

\hypertarget{toc9}{}
\subsubsection{Web server access (Apache, Nginx, Cherokee)}

The main idea of sending information to SaltOS is to use the latest technologies, to do it, we are using restful request

\begin{compactitem}
\item[\color{myblue}$\bullet$] GET with REST:
  \begin{compactitem}
  \item[\color{myblue}$\bullet$] An example request: `\htmladdnormallink{https://host/?/app/invoices/view/2}{https://host/?/app/invoices/view/2}`
  \item[\color{myblue}$\bullet$] `@rest/0` = app
  \item[\color{myblue}$\bullet$] `@rest/1` = invoices
  \item[\color{myblue}$\bullet$] `@rest/2` = view
  \item[\color{myblue}$\bullet$] `@rest/3` = 2
  \end{compactitem}
\item[\color{myblue}$\bullet$] POST with JSON:
  \begin{compactitem}
  \item[\color{myblue}$\bullet$] An example request: `\htmladdnormallink{https://host/?/app/invoices/insert}{https://host/?/app/invoices/insert}`
  \item[\color{myblue}$\bullet$] Use `Content-Type: application/json`
  \item[\color{myblue}$\bullet$] Body parsed into `@json/...`
  \end{compactitem}
\end{compactitem}

\hypertarget{toc10}{}
\subsubsection{Command-line access (CLI)}

\begin{compactitem}
\item[\color{myblue}$\bullet$] `php api/index.php app/customers/view/100`
\item[\color{myblue}$\bullet$] `user=admin php api/index.php app/customers/view/100`
\item[\color{myblue}$\bullet$] `cat data.json $|$ user=admin php app/customers/insert`
\end{compactitem}


\hypertarget{toc11}{}
\section{Frontend Structure (web/)}

Frontend is a JavaScript SPA in the `web/` folder.

\begin{compactitem}
\item[\color{myblue}$\bullet$] `index.htm`: loads Bootstrap, SaltOS scripts
\item[\color{myblue}$\bullet$] `web/js/`: client-side modules
\item[\color{myblue}$\bullet$] `web/lib/`: JS libraries
\end{compactitem}

JavaScript Modules:

\begin{compactitem}
\item[\color{myblue}$\bullet$] `app.js`: * Application helper module
\item[\color{myblue}$\bullet$] `auth.js`: * Authentication helper module
\item[\color{myblue}$\bullet$] `backup.js`: * Backup \& Autosave helper module
\item[\color{myblue}$\bullet$] `bootstrap.js`: * Bootstrap helper module
\item[\color{myblue}$\bullet$] `common.js`: * Common helper module
\item[\color{myblue}$\bullet$] `core.js`: * Core helper module
\item[\color{myblue}$\bullet$] `driver.js`: * Driver module
\item[\color{myblue}$\bullet$] `filter.js`: * Filter module
\item[\color{myblue}$\bullet$] `form.js`: * Form helper module
\item[\color{myblue}$\bullet$] `gettext.js`: * Gettext helper module
\item[\color{myblue}$\bullet$] `hash.js`: * Hash helper module
\item[\color{myblue}$\bullet$] `object.js`: * Object helper module
\item[\color{myblue}$\bullet$] `proxy.js`: * Proxy module
\item[\color{myblue}$\bullet$] `push.js`: * Push \& favicon helper module
\item[\color{myblue}$\bullet$] `storage.js`: * Token helper module
\item[\color{myblue}$\bullet$] `token.js`: * Token helper module
\item[\color{myblue}$\bullet$] `window.js`: * Window helper module
\end{compactitem}

\hypertarget{toc12}{}
\subsection{Frontend Deployment}

To use SaltOS4 from the browser:

\begin{compactitem}
\item[\color{myblue}$\bullet$] Publish `web/`
\item[\color{myblue}$\bullet$] Create symbolic links inside `web/`:
  \begin{compactitem}
  \item[\color{myblue}$\bullet$] `apps/` → application resources
  \item[\color{myblue}$\bullet$] `api/` → backend
  \end{compactitem}
\item[\color{myblue}$\bullet$] Inside `api/`, symbolic links to:
  \begin{compactitem}
  \item[\color{myblue}$\bullet$] `data/` → file storage
  \item[\color{myblue}$\bullet$] `apps/` → app definitions
  \end{compactitem}
\end{compactitem}


\hypertarget{toc13}{}
\section{Application System}

Applications in SaltOS4 are modular, defined in folders under `apps/`.

Each folder may contain:

\begin{compactitem}
\item[\color{myblue}$\bullet$] `xml/manifest.xml`: declares the apps
\item[\color{myblue}$\bullet$] `xml/*.xml` or `xml/*.yaml`: application definitions
\item[\color{myblue}$\bullet$] `php/`, `js/`, `locale/`: optional logic and translations
\item[\color{myblue}$\bullet$] `dbschema.xml`, `dbstatic.xml`: database structure and static content
\end{compactitem}

\hypertarget{toc14}{}
\subsection{YAML-based Apps}

SaltOS4 supports YAML for fast app declarations using the common template logic.

\hypertarget{toc15}{}
\subsubsection{tokenslog.yaml}

\begin{lstlisting}
app: tokenslog
require: apps/common/php/default.php
template: apps/common/xml/default.xml
indent: true
screen: type2
list:
    # [id, type, label]
    - [user_id, select, User]
    - [created_at, text, Created]
    - [token, text, Token]
    - [active, boolean, Active]
form:
    # [id, type, label]
    - [active, switch, Active]
    - [user_id, select, User]
    - [created_at, datetime, Created]
    - [updated_at, datetime, Updated]
    - [remote_addr, text, Remote Addres]
    - [user_agent, text, User Agent]
    - [token, text, Token]
    - [expires_at, datetime, Expires]
select:
    # [id, table, optional field]
    - [user_id, tbl_users]
\end{lstlisting}

\hypertarget{toc16}{}
\subsubsection{configlog.yaml}

\begin{lstlisting}
app: configlog
require: apps/common/php/default.php
template: apps/common/xml/default.xml
indent: true
screen: type2
list:
    # [id, type, label]
    - [user_id, select, User]
    - [key, text, Key]
form:
    # [id, type, label]
    - [user_id, select, User]
    - [key, text, Key]
    - [val, codemirror, Value]
select:
    # [id, table, optional field]
    - [user_id, tbl_users]
attr:
    # field:
    #   attr: value
    key:
        required: true
        autofocus: true
    val:
        required: true
        mode: json
        indent: true
\end{lstlisting}

\hypertarget{toc17}{}
\subsubsection{types.yaml}

\begin{lstlisting}
app: types
require: apps/common/php/default.php
template: apps/common/xml/default.xml
indent: true
screen: type5
list:
    # [id, type, label]
    - [name, text, Name]
    - [description, text, Description]
    - [active, boolean, Active]
form:
    # [id, type, label]
    - [active, switch, Active]
    - [name, text, Name]
    - [description, textarea, Description]
attr:
    # field:
    #   attr: value
    name:
        required: true
        autofocus: true
    description:
        required: true
\end{lstlisting}

These examples demonstrate the simplicity of defining apps in YAML.

\hypertarget{toc18}{}
\subsection{XML-based Complex Apps}

SaltOS4 also allows full custom apps defined in XML, often used for more complex business logic and interface customization.

\hypertarget{toc19}{}
\subsubsection{customers.xml}

This file contains the follow important nodes:

\begin{compactitem}
\item[\color{myblue}$\bullet$] `$<$main$>$`: Defines a call to app/customers/main that returns the screen type, literals, and the navbar specification
\item[\color{myblue}$\bullet$] `$<$list default="true"$>$`: Defines a call to app/customers/list that returns a cache load from app/customers/list/cache
\item[\color{myblue}$\bullet$] `$<$list id="cache"$>$`: Defines a call to app/customers/list/cache that returns the interface of a list screen
\item[\color{myblue}$\bullet$] `$<$list id="data"$>$`: Defines a call to app/customers/list/data that returns the data of a list
\item[\color{myblue}$\bullet$] `$<$\_form$>$`: This defines a form, there is no way to access it externally, it is for internal use
\item[\color{myblue}$\bullet$] `$<$\_data require="php/lib/log.php" eval="true"$>$`: This defines a data block of a detail view, not accessible externally, intended for internal use
\item[\color{myblue}$\bullet$] `$<$create$>$`: Defines a call to app/customers/create that returns a cache load from app/customers/create/cache
\item[\color{myblue}$\bullet$] `$<$create id="cache"$>$`: Defines a call to app/customers/create/cache that returns the interface of a creation screen
\item[\color{myblue}$\bullet$] `$<$create id="insert"$>$`: Defines a call to app/customers/insert that allows inserting a record and returns the status
\item[\color{myblue}$\bullet$] `$<$view$>$`: Defines a call to app/customers/view that returns a cache load from app/customers/view/cache
\item[\color{myblue}$\bullet$] `$<$view id="cache"$>$`: Defines a call to app/customers/view/cache that returns the interface of a creation screen
\item[\color{myblue}$\bullet$] `$<$edit$>$`: Defines a call to app/customers/edit that returns a cache load from app/customers/edit/cache
\item[\color{myblue}$\bullet$] `$<$edit id="cache"$>$`: Defines a call to app/customers/edit/cache that returns the interface of a creation screen
\item[\color{myblue}$\bullet$] `$<$edit id="update"$>$`: Defines a call to app/customers/update that allows updating a record and returns the status
\item[\color{myblue}$\bullet$] `$<$delete$>$`: Defines a call to app/customers/delete that allows deleting a record and returns the status
\item[\color{myblue}$\bullet$] `$<$action id="setup"$>$`: Defines a call to app/customers/setup that allows to execute the setup for this application
\end{compactitem}


\hypertarget{toc20}{}
\section{Offline Support (Service Worker)}

Handled in `proxy.js` + `core.js`:

\begin{compactitem}
\item[\color{myblue}$\bullet$] Automatically registers in the browser
\item[\color{myblue}$\bullet$] Acts as a proxy for fetch requests
\item[\color{myblue}$\bullet$] Returns cached responses if offline
\item[\color{myblue}$\bullet$] Queues write requests and syncs when online
\item[\color{myblue}$\bullet$] Works transparently with core AJAX logic
\end{compactitem}


\hypertarget{toc21}{}
\section{Makefile Overview}

This `Makefile` automates building, testing, documentation, and setup tasks in SaltOS4.

\hypertarget{toc22}{}
\subsection{Build Targets}

\begin{compactitem}
\item[\color{myblue}$\bullet$] `make web`: Builds and minifies production assets:
  \begin{compactitem}
  \item[\color{myblue}$\bullet$] Combines and compresses CSS/JS
  \item[\color{myblue}$\bullet$] Uses `fixpath.php`, `md5sum.php`, `uglifyjs`, `minify`, `sha384.php`
  \item[\color{myblue}$\bullet$] Handles app-specific JS from `apps/*/js/*.js`
  \item[\color{myblue}$\bullet$] Generates the `proxy.js` script with source maps
  \end{compactitem}
\item[\color{myblue}$\bullet$] `make devel`: Prepares a development environment with unminified assets using `debug.php`.
\item[\color{myblue}$\bullet$] `make clean`: Deletes generated files:
  \begin{compactitem}
  \item[\color{myblue}$\bullet$] Minified JS, CSS, maps, HTML, `proxy.js`, and per-app compiled assets
  \end{compactitem}
\end{compactitem}

\hypertarget{toc23}{}
\subsection{Documentation}

\begin{compactitem}
\item[\color{myblue}$\bullet$] `make docs`: Generates `.t2t`, `.pdf`, and `.html` documentation using scripts:
  \begin{compactitem}
  \item[\color{myblue}$\bullet$] Backend: `docs/api.t2t`
  \item[\color{myblue}$\bullet$] Frontend: `docs/web.t2t`
  \item[\color{myblue}$\bullet$] Applications: `docs/apps.t2t`
  \item[\color{myblue}$\bullet$] PHP tests: `docs/utest.t2t`
  \item[\color{myblue}$\bullet$] JS tests: `docs/ujest.t2t`
  \item[\color{myblue}$\bullet$] Developer manual: `docs/devel.t2t`
  \end{compactitem}
\end{compactitem}

\hypertarget{toc24}{}
\subsection{Testing}

\begin{compactitem}
\item[\color{myblue}$\bullet$] `make test`: Runs:
  \begin{compactitem}
  \item[\color{myblue}$\bullet$] `phpcs`, `php -l`, `phpstan` on PHP files
  \item[\color{myblue}$\bullet$] `jscs`, `node -c` on JS files
  \item[\color{myblue}$\bullet$] Accepts variable `file=...`, `file=all`, or none (uses SVN diff)
  \end{compactitem}
\item[\color{myblue}$\bullet$] `make utest`: Runs PHPUnit with optional filtering by file
\item[\color{myblue}$\bullet$] `make ujest`: Runs JS tests with Jest:
  \begin{compactitem}
  \item[\color{myblue}$\bullet$] Cleans diff snapshots and temp coverage
  \item[\color{myblue}$\bullet$] Supports full or filtered test runs
  \item[\color{myblue}$\bullet$] Generates coverage report
  \end{compactitem}
\end{compactitem}

\hypertarget{toc25}{}
\subsection{Environment Checks}

\begin{compactitem}
\item[\color{myblue}$\bullet$] `make check`: Verifies required folders and system commands:
  \begin{compactitem}
  \item[\color{myblue}$\bullet$] Symbolic links: `api/data`, `web/apps`, etc.
  \item[\color{myblue}$\bullet$] Commands: `php`, `node`, `jest`, `phpunit`, `uglifyjs`, `txt2tags`, etc.
  \end{compactitem}
\end{compactitem}

\hypertarget{toc26}{}
\subsection{Dependency Management}

\begin{compactitem}
\item[\color{myblue}$\bullet$] `make libs`: Checks required PHP libraries via `checklibs.php`
\end{compactitem}

\hypertarget{toc27}{}
\subsection{Code Statistics}

\begin{compactitem}
\item[\color{myblue}$\bullet$] `make cloc`: Uses `cloc` to count lines of code excluding minified and ignored files
\end{compactitem}

\hypertarget{toc28}{}
\subsection{System Setup}

\begin{compactitem}
\item[\color{myblue}$\bullet$] `make setup`: Initializes SaltOS4 system
\item[\color{myblue}$\bullet$] `make setupmysql`: Sets up all applications using MariaDB
\item[\color{myblue}$\bullet$] `make setupsqlite`: Sets up all applications using SQLite
\item[\color{myblue}$\bullet$] `make setupinstall`: Combines cleaning + full MySQL + SQLite setup
\item[\color{myblue}$\bullet$] `make setupclean`: Cleans data directories and resets the database
\end{compactitem}

\hypertarget{toc29}{}
\subsection{System Actions}

\begin{compactitem}
\item[\color{myblue}$\bullet$] `make gc`: Launches garbage collection
\item[\color{myblue}$\bullet$] `make indexing`: Performs indexing operations
\item[\color{myblue}$\bullet$] `make integrity`: Runs integrity checks
\item[\color{myblue}$\bullet$] `make cron`: Executes scheduled tasks (cron)
\end{compactitem}


\hypertarget{toc30}{}
\section{Script Directory Overview}

This section describes the purpose of each script and configuration file in the `scripts/` directory.

All descriptions below are based on the actual content of each file.

\begin{itemize}
\item[\color{myblue}$\bullet$] `checklibs.php`: Validates the current versions of required libraries by parsing `checklibs.txt`, performing curl requests, and comparing base64-encoded version strings. Updates the file if needed.
\item[\color{myblue}$\bullet$] `checklibs.txt`: Contains a list of required libraries, with their URLs and expected version tags encoded in base64.
\item[\color{myblue}$\bullet$] `debug.php`: Generates a debug-friendly version of the frontend using `debug.php`.
\item[\color{myblue}$\bullet$] `fixpath.php`: Adjusts file paths in generated HTML/JS/CSS to match the deployment structure.
\item[\color{myblue}$\bullet$] `jest.config.js`: Configuration file for running JavaScript unit tests with Jest.
\item[\color{myblue}$\bullet$] `jest\_coverage.php`: Processes Jest coverage reports and outputs them in a readable format.
\item[\color{myblue}$\bullet$] `jest\_tester.php`: Parses the layout structure from `apps/tester/xml/tester.xml`, checks for duplicate widget tags, and exports the data as `/tmp/tester.json`.
\item[\color{myblue}$\bullet$] `jscs.json`: Defines JavaScript coding standards used by the JSCS code style checker.
\item[\color{myblue}$\bullet$] `make\_bootswatch.php`: Cleans up Bootswatch CSS files by removing any external `@import` statements.
\item[\color{myblue}$\bullet$] `make\_instance.sh`: Creates a new runnable SaltOS4 instance by linking `.htaccess`, preparing folders like `data/` and `tmp/`, and applying permissions.
\item[\color{myblue}$\bullet$] `makehtml.php`: Converts a `.t2t` documentation file into an HTML file using `txt2tags`.
\item[\color{myblue}$\bullet$] `maket2t.php`: Scans a source directory and extracts `/** ... */` comments from each file to generate a `.t2t` documentation file.
\item[\color{myblue}$\bullet$] `makepdf.php`: Converts a `.t2t` documentation file into a `.pdf` using LaTeX.
\item[\color{myblue}$\bullet$] `md5sum.php`: Generates an MD5 checksum of one or more files.
\item[\color{myblue}$\bullet$] `migrate\_v3\_to\_v4.php`: Migrates a SaltOS system from version 3 to version 4, adapting data and file structure.
\item[\color{myblue}$\bullet$] `phpcs.xml`: Configuration file for PHP\_CodeSniffer to enforce PHP coding standards.
\item[\color{myblue}$\bullet$] `phpstan.neon`: Configuration file for PHPStan, specifying analysis rules and paths for static code analysis.
\item[\color{myblue}$\bullet$] `phpunit.xml`: Configuration file for PHPUnit, specifying test directories, filters, and bootstrap files.
\item[\color{myblue}$\bullet$] `sha384.php`: Calculates SHA-384 hashes for files to use in Subresource Integrity (SRI) attributes in HTML.
\item[\color{myblue}$\bullet$] `updatet2t.php`: Updates the second and third lines of a `.t2t` file (used to update the version and date of `devel.t2t`).

\end{itemize}

% LaTeX2e code generated by txt2tags 3.4 (http://txt2tags.org)
% cmdline: txt2tags --toc -t tex -i devel.t2t -o devel.tex
\end{document}
